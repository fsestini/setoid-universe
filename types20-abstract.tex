\documentclass{easychair}

\usepackage{amsmath}
\usepackage{amssymb}
\usepackage{mathpartir}
\usepackage{proof}
\usepackage{tikz}
\usepackage{bbm}
\usepackage[bw]{agda}
\usetikzlibrary{cd}

\newcommand{\setoidU}{\mathcal{U}}
\newcommand{\ad}[1]{\AgdaFunction{#1}}
\newcommand{\Set}{\textsf{Set}}
\newcommand{\Prop}{\textsf{Prop}}
\newcommand{\Setoid}{\textsf{Setoid}}
\newcommand{\Con}{\mathrm{Con}}
\newcommand{\Ty}{\mathrm{Ty}}
\newcommand{\Tm}{\mathrm{Tm}}
\newcommand{\Tms}{\mathrm{Tms}}
\newcommand{\id}{\mathrm{id}}
\newcommand{\op}{\mathrm{op}}
\newcommand{\cat}[1]{\underline{\mathbf{#1}}}
\newcommand{\U}{\textsf{U}}
\newcommand{\J}{\mathrm{J}}
\newcommand{\El}{\textsf{El}}

% \newcommand{\Set}{\mathbf{Set}}
% \newcommand{\Prop}{\mathbf{Prop}}
\newcommand{\Type}{\mathbf{Type}}
\renewcommand{\iff}{\leftrightarrow}
\newcommand{\Nat}{\mathbb{N}}
\newcommand{\Int}{\mathbb{Z}}
\newcommand{\Rat}{\mathbb{Q}}
\newcommand{\Real}{\mathbb{R}}
\newcommand{\Bool}{\mathrm{Bool}}

\newcommand{\List}{\mathrm{List}}
\newcommand{\rev}{\mathrm{rev}}

%\newtheorem{theorem}{Theorem}

\providecommand\mathbbm{\mathbb}
\def\lv{\mathopen{{[\kern-0.14em[}}}    % opening [[ value delimiter
\def\rv{\mathclose{{]\kern-0.14em]}}}   % closing ]] value delimiter
\newcommand{\eval}[1]{{\lv{#1}\rv}}

% \usepackage{amsmath}
% \usepackage{amssymb}
% %\usepackage{mathpartir}

% \newcommand{\Set}{\mathbf{Set}}
% \newcommand{\Prop}{\mathbf{Prop}}
% \newcommand{\Type}{\mathbf{Type}}
% \renewcommand{\iff}{\leftrightarrow}
% \newcommand{\Nat}{\mathbb{N}}
% \newtheorem{theorem}{Theorem}
% \newcommand{\cont}[2]{#1 \lhd #2}
% \def\lv{\mathopen{{[\kern-0.14em[}}}    % opening [[ value delimiter
% \def\rv{\mathclose{{]\kern-0.14em]}}}   % closing ]] value delimiter
% \newcommand{\eval}[1]{{\lv{#1}\rv}}

%\usepackage{natbib}
%\setcitestyle{round}
% Note that the \doi command from the doi package doesn't enable the
% same kinds of line breaks as the command below.
\newcommand{\doi}[1]{doi:\href{http://doi.org/#1}{%
    \urlstyle{same}\nolinkurl{#1}}}

% TODO notes.
\usepackage[textsize=small]{todonotes}
\setlength{\marginparwidth}{2cm}
\newcommand{\txatodo}[1]{\todo[fancyline,color=green!20]{#1}{}}

\title{Constructing a universe for the setoid model}

\author{Thorsten Altenkirch\inst{1}\thanks{Supported by USAF grant
    FA9550-16-1-0029.} \and AmbrusKaposi \inst{2} \and Filippo Sestini\inst{1}}

\institute{
  School of Computer Science, University of Nottingham, UK\\
  \email{\{psztxa,psxfs5\}@nottingham.ac.uk}
  \and
  E{\"o}tv{\"o}s Lor{\'a}nd University, Budapest, Hungary\\
  \email{akaposi@inf.elte.hu}
}

\begin{document}
\maketitle

The setoid interpretation gives rise to a model of a type theory with functional
and propositional extensionality. It is thus a way to explain extensionality in
a type-theoretic and computationally adequate way \cite{alti99}. The setoid
model has been recently adapted into Setoid Type Theory (SeTT), which is
justified by a syntactic translation into a very basic \emph{target theory}.
\cite{mpc19}. This translation relies on the existence in the target theory of a
universe of definitionally proof-irrelevant propositions $\Prop$, as recently
implemented in Coq and Agda \cite{Sprop}. The setoid interpretation models a
universe of propositions where equality of propositions is logical equivalence,
thus providing a very basic instance of Univalence. However, we also would like
to be able to equip SeTT with a universe of setoids. This universe must in turn
be a setoid, hence in particular it cannot be univalent, but rather its
equivalence relation must reflect simple equality of codes.

To provide such a universe we need to define a setoid $\setoidU : \Setoid$ and a
family of setoids $\El : \setoidU \to \Setoid$, with codes for basic type
formers like $\Pi$-types and booleans. This can be obtained as an inductive
recursive type.
%
\begin{align*}
  & \textsf{data} \ \setoidU : \Set_1 \\
  & \_\sim_\setoidU\_ : \setoidU \to \setoidU \to \Prop_1 \\
  & \El : \setoidU \to \Set \\
  & \_\vdash\_\sim_\El\_ : \{a\ a' : \setoidU\} \to a \sim_\setoidU a' \to \El\ a \to \El\ a' \to \Prop
\end{align*}

However, we don't want to assume inductive-recursive types in the basic type
theory which is the target of the setoid model. We know that we can translate
basic instances of induction-recursion into inductive families using the
equivalence of families and discrete fibrations. For example, the
inductive-recursive definition of a universe $\U:\Set_1$ and $\El : \U \to
\Set$ with $\Pi$-types and booleans is
\footnote{In Agda, this definition would also go through if $\U$ were in $\Set$, but
  this seems to be a non-conservative extension.}:
%
\begin{align*}
  & \textsf{data}\ \U\ : \Set_1 \\
  & \quad \textsf{bool} : \U \\
  & \quad \textsf{pi} : (A : \U) \to (\El\ A \to \U) \to \U \\
  & \El\ \textsf{bool} = \mathbbm{2} \\
  & \El\ (\textsf{pi}\ A\ B) = (a : \El\ A) \to \El\ (B\ a)
\end{align*}

We can model this as an inductive type $\ad{in-U}$ that \emph{carves out} all
types in $\Set$ that are in the image of $\El$:
%
\begin{align*}
  & \textsf{data}\ \ad{in-U} : \Set \to \Set_1 \\
  & \quad \textsf{inBool} : \textsf{in-U}\ \mathbbm{2} \\
  & \quad \textsf{inPi}
  :  \{A : \Set\} \{B : A \to \Set\}\\
  & \ \qquad \to \textsf{in-U}\ A
  \to ((a : A) \to \textsf{in-U}\ (B\ a))
  \to \textsf{in-U}\ ((a : A) \to (B\ a))
\end{align*}
\begin{align*}
  \U & = \Sigma (A : \Set) (\textsf{in-U}\ A) \\
  \El & = \textsf{proj}_1
\end{align*}

Note that this construction gives rise to an inductive-recursive universe in
$\Set_1$, rather than $\Set$, thus the meta-theory must contain at least one
universe. Our result is that a modified form of this translation also works for
the more complex inductive-recursive type we need to model the universe of
setoids. 
%
In particular, in addition to $\ad{in-U}$ for defining $\U$ as before, we also
introduce a family $\ad{in-U∼}$ of binary relations between types in the
universe, from which we then define $\_\ad{∼U}\_$.
%
However, we now need an inductive-inductive type in the target theory.
%
\begin{align*}
  & \textsf{data}\ \ad{in-U} : \Set \to \Set_1 \\
  & \textsf{data}\ \ad{in-U∼} : \{A\ A' : \Set\} \to \ad{in-U}\ A \to \ad{in-U}\ A' \to (A \to A' \to \Prop) \to \Set_1 \\
  & \setoidU : \Set_1 \\
  & \El : \setoidU \to \Set \\
  & \_\sim_{\setoidU}\_ : \setoidU \to \setoidU \to \Prop_1 \\
  & \_\vdash\_\sim_{\El}\_ : \{a\ a' : \U\} \to a\ \sim_{\setoidU}\ a' \to \El\ a \to \El\ a' \to \Prop \\
  & \\
  & \U = \Sigma\ (X : \Set)\ (\AgdaDatatype{in-U}\ X) \\
  & \El = \textsf{proj}_1 \\
  & (X , p) \sim_{\setoidU} (X' , p') =
   \ \parallel \Sigma\ (R : X \to X' \to \Prop) (\ad{in-U~} p\ p'\ R) \parallel
\end{align*}

where $\parallel A \parallel\ : \Prop$ is propositional truncation for any $A :
\Set$. Intuition would suggest to define $\_\vdash\_\sim_{\El}\_$ by projecting
the relation out of the proof of $a \sim_{\setoidU} a'$, in much the same way as
$\El$ is defined by projecting out of $\U$. That is, $p \vdash x \sim_{\El} y =
\ad{proj}_1\ p\ x\ y$.
%
However, this doesn't work since the type of $p$ is propositionally truncated,
hence it cannot be used to construct a proof-relevant object. We can work around
this by defining $\_\vdash\_\sim_{\El}\_$ explicitly for all type formers, by
induction on the codes $a\ a' : \U$, mutually with a proof that this definition
is in some sense equivalent to what we would get if we could project out of the
truncation.

We know that finitary inductive-inductive definitions can be translated into
inductive families \cite{iit-erasure,iit-to-ix,induction-is-enough} but it is
not clear whether this construction extends to an infinitary type like the one
above. This is subject of further work. If successful we would be able to give a
translation of the setoid model with a universe into a very basic core type
theory.

\bibliography{bibliography}

\end{document}
