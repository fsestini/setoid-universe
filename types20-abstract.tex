\documentclass{easychair}

\usepackage{amsmath}
\usepackage{amssymb}
\usepackage{mathpartir}
\usepackage{proof}
\usepackage{tikz}
\usetikzlibrary{cd}


\newcommand{\Set}{\mathbf{Set}}
\newcommand{\Prop}{\mathbf{Prop}}
\newcommand{\Setoid}{\mathbf{Setoid}}
\newcommand{\Cat}{\mathbf{Cat}}
\newcommand{\Con}{\mathrm{Con}}
\newcommand{\Ty}{\mathrm{Ty}}
\newcommand{\Tm}{\mathrm{Tm}}
\newcommand{\Tms}{\mathrm{Tms}}
\newcommand{\id}{\mathrm{id}}
\newcommand{\op}{\mathrm{op}}
\newcommand{\cat}[1]{\underline{\mathbf{#1}}}
\newcommand{\U}{\mathrm{U}}
\newcommand{\J}{\mathrm{J}}
\newcommand{\El}{\mathrm{El}}

% \newcommand{\Set}{\mathbf{Set}}
% \newcommand{\Prop}{\mathbf{Prop}}
\newcommand{\Type}{\mathbf{Type}}
\renewcommand{\iff}{\leftrightarrow}
\newcommand{\Nat}{\mathbb{N}}
\newcommand{\Int}{\mathbb{Z}}
\newcommand{\Rat}{\mathbb{Q}}
\newcommand{\Real}{\mathbb{R}}
\newcommand{\Bool}{\mathrm{Bool}}

\newcommand{\List}{\mathrm{List}}
\newcommand{\rev}{\mathrm{rev}}

%\newtheorem{theorem}{Theorem}

\providecommand\mathbbm{\mathbb}
\def\lv{\mathopen{{[\kern-0.14em[}}}    % opening [[ value delimiter
\def\rv{\mathclose{{]\kern-0.14em]}}}   % closing ]] value delimiter
\newcommand{\eval}[1]{{\lv{#1}\rv}}

% \usepackage{amsmath}
% \usepackage{amssymb}
% %\usepackage{mathpartir}

% \newcommand{\Set}{\mathbf{Set}}
% \newcommand{\Prop}{\mathbf{Prop}}
% \newcommand{\Type}{\mathbf{Type}}
% \renewcommand{\iff}{\leftrightarrow}
% \newcommand{\Nat}{\mathbb{N}}
% \newtheorem{theorem}{Theorem}
% \newcommand{\cont}[2]{#1 \lhd #2}
% \def\lv{\mathopen{{[\kern-0.14em[}}}    % opening [[ value delimiter
% \def\rv{\mathclose{{]\kern-0.14em]}}}   % closing ]] value delimiter
% \newcommand{\eval}[1]{{\lv{#1}\rv}}

%\usepackage{natbib}
%\setcitestyle{round}
% Note that the \doi command from the doi package doesn't enable the
% same kinds of line breaks as the command below.
\newcommand{\doi}[1]{doi:\href{http://doi.org/#1}{%
    \urlstyle{same}\nolinkurl{#1}}}

% TODO notes.
\usepackage[textsize=small]{todonotes}
\setlength{\marginparwidth}{2cm}
\newcommand{\txatodo}[1]{\todo[fancyline,color=green!20]{#1}{}}

\title{Constructing a universe for the setoid model}

\author{Thorsten Altenkirch\inst{1}\thanks{Supported by USAF grant
    FA9550-16-1-0029.} \and AmbrusKaposi \inst{2} \and Filippo Sestini\inst{1}}

\institute{
  School of Computer Science, University of Nottingham, UK\\
  \email{\{psztxa,psxfs5\}@nottingham.ac.uk}
  \and
  E{\"o}tv{\"o}s Lor{\'a}nd University, Budapest, Hungary\\
  \email{akaposi@inf.elte.hu}
}

\begin{document}
\maketitle

The setoid model gives rise to a translation of a type theory with
functional and propositional extensionality to one without. It is thus
a way to explain extensionality in a type-theoretic and
computationally adequate way \cite{alti99,mpc19}. This translation
relies on the existence of a strict universe of propositions $\Prop$
as implemented in Coq and Agda \cite{Sprop}. This interpretation
in turn models a universe of propositions where equality of propositions
is logical equivalence, this is a very basic instance of
univalence. However, we also would like to be able to refer to a
universe of setoids, which cannot be univalent, because this would not
be a setoid. 

To provide such a universe we need $\U : \Setoid$ and a family of
setoids $\El : \U \to \Setoid$ which provide codes for basic type
formers, like $\Pi$-types and booleans. This can be obtained as an
inductive recursive type.

\emph{Insert a summary of the construction}

However, we don't want to assume inductive-recursive types in the
basic type theory which is the target of the setoid model. We know
that we can translate basic instances of induction-recursion into
inductive families using the equivalence of families and discrete
fibrations. E.g. the inductive recursive definition of a universe $\U:\Set$
and $\El : \U \to \Setoid$ with $\Pi$-types and booleans is:

\emph{insert definition}

Assuming a universe of sets in the target theory we can model this as
an inductive type as follows:

\emph{insert definition}

Our result is that a modified form of this translation also works for
the more complex type we need to model the setoid universe. However,
we now need an inductive-inductive type in the target theory.

\emph{Insert a summary of the construction}

We know that finitary indutive-inductive definitions can be translated
into inductive families \cite{jakob,ambroise} but it is not clear wether this construction
extends to an infinitary type like the one above. This is subject of
further work. If successful we would be able to give a translation of
the setoid model with a universe into a very basic core type theory.

\bibliographystyle{alpha}
\bibliography{local}

\end{document}
